\documentclass[11pt]{article}
\usepackage[utf8]{inputenc}
\usepackage[spanish]{babel}
\usepackage{amsmath, amssymb}
\usepackage{graphicx}
\usepackage{hyperref}
\usepackage{geometry}
\usepackage{fancyhdr}
\usepackage{listings}
\usepackage{xcolor}
\geometry{a4paper, margin=2.5cm}

\lstdefinestyle{mystyle}{
  backgroundcolor=\color{gray!10},   % fondo gris claro
  commentstyle=\color{green!50!black}, 
  keywordstyle=\color{blue},
  numberstyle=\tiny\color{gray},
  stringstyle=\color{red!70!brown},
  basicstyle=\ttfamily\small,        % fuente monoespaciada más pequeña
  breaklines=true,                   
  captionpos=b,
  keepspaces=true,
  numbers=left,                      % numeración de líneas a la izquierda
  numbersep=5pt,                     
  showspaces=false,
  showstringspaces=false,
  showtabs=false,
  frame=single,                      % cuadro alrededor del código
  language=Matlab
}

\lstset{style=mystyle}



\title{\Huge Sistemas de Control II \\[0.4cm] \Large Tarea 1}
\date{}

\begin{document}

% Carátula
\begin{titlepage}
    \centering
    \vspace*{2cm}
    {\Huge\bfseries Sistemas de Control II \par}
    \vspace{0.5cm}
    {\Large Tarea 1 \par}
    \vspace{2cm}
    \begin{flushleft}
        \large
        \textbf{Profesor:} Sergio Laboret\\[0.3cm]
        \textbf{Alumno:} Angeloff Jorge
    \end{flushleft}
    \vspace{2cm}
    \begin{center}
        \begin{tabular}{|c|c|c|c|}
            \hline
            Polo 1 & Polo 2 & Cero & Ganancia \\
            \hline
            0 & -2 & -- & 5 \\
            \hline
        \end{tabular}
    \end{center}
    \vfill
    {\large \today}
\end{titlepage}

\section{Análisis de un sistema analógico con muestreador y retentor de orden cero}

\subsection{Obtención de la función de transferencia continua \(G(s)\)}
\begin{lstlisting}[language=Matlab]
G = zpk([], [0 -2], [5]);
Tm = 0.15;
\end{lstlisting}

Se define el sistema continuo como una función de transferencia en forma de ceros, polos y ganancia, con polos en \(s = 0\) y \(s = -2\), sin ceros, y ganancia \(K = 5\). El tiempo de muestreo se establece en 0.15 segundos.

\begin{figure}[h!]
    \centering
    \includegraphics[width=0.5\textwidth]{/home/jorge/Documentos/FCEFyN/SC2/Laboret/Tarea1/capturas/figura1.png}
    \caption{Definición de la función de transferencia continua}
\end{figure}

\subsection{Transformación a un sistema discreto con retención ZOH}
\begin{lstlisting}[language=Matlab]
Gd = c2d(G, Tm, 'zoh');
\end{lstlisting}

La transformación discreta con retención de orden cero permite obtener la función de transferencia \(G_d(z)\) del sistema. Esto representa al sistema cuando es muestreado con periodo \(T_m\).

\begin{figure}[h!]
    \centering
    \includegraphics[width=0.3\textwidth]{/home/jorge/Documentos/FCEFyN/SC2/Laboret/Tarea1/capturas/figura2.png}
    \caption{Funcion de transferencia discreta obtenida mediante ZOH}
\end{figure}

\subsection{Mapa de polos y ceros del sistema continuo y discreto}
\begin{lstlisting}[language=Matlab]
pzmap(G);
pzmap(Gd);
\end{lstlisting}

Comparar los mapas de polos y ceros del sistema continuo y su contraparte discreta revela cómo el muestreo afecta la localización de los polos.

\begin{figure}[h!]
    \centering
    \includegraphics[width=0.48\textwidth]{/home/jorge/Documentos/FCEFyN/SC2/Laboret/Tarea1/capturas/figura3_1.png}
    \includegraphics[width=0.48\textwidth]{/home/jorge/Documentos/FCEFyN/SC2/Laboret/Tarea1/capturas/figura3_2.png}
    \caption{Mapas de polos y ceros del sistema continuo (izq.) y discreto (der.)}
\end{figure}

\newpage

\subsection{Efecto de aumentar el periodo de muestreo}
\begin{lstlisting}[language=Matlab]
Gd1 = c2d(G, 10*Tm, 'zoh');
pzmap(Gd1);
\end{lstlisting}

Al aumentar el tiempo de muestreo, los polos del sistema discreto se acercan al origen del plano \(z\), lo cual reduce la velocidad de respuesta del sistema.

\begin{figure}[h!]
    \centering
    \includegraphics[width=0.3\textwidth]{/home/jorge/Documentos/FCEFyN/SC2/Laboret/Tarea1/capturas/figura4_1.png}
    \includegraphics[width=0.6\textwidth]{/home/jorge/Documentos/FCEFyN/SC2/Laboret/Tarea1/capturas/figura4_2.png}
    \caption{Sistema con \(T_m\) aumentado. Funcion de transferencia (izq.), mapa de polos y cero (der.)}
\end{figure}

\subsection{Respuesta al escalón y análisis de estabilidad}
\begin{lstlisting}[language=Matlab]
step(G)
step(Gd1)
\end{lstlisting}

La respuesta discreta con \(Gd1\) muestra una rampa creciente: el sistema es marginalmente estable, pero no acotado.

\begin{figure}[h!]
    \centering
    \includegraphics[width=0.48\textwidth]{/home/jorge/Documentos/FCEFyN/SC2/Laboret/Tarea1/capturas/figura5_1.png}
    \includegraphics[width=0.48\textwidth]{/home/jorge/Documentos/FCEFyN/SC2/Laboret/Tarea1/capturas/figura5_2.png}
    \caption{Respuesta al escalón de sistema continuo (izq.) y discreto \(Gd1\) (der.)}
\end{figure}

\newpage

\subsection{Tipo de sistema y error en estado estacionario}
\begin{lstlisting}[language=Matlab]
Kp = dcgain(Gd)
F = feedback(Gd,1)
step(F)
ess = 1 / (1 + Kp)
\end{lstlisting}

\(G_d(z)\) es de tipo 1 por su polo en \(z = 1\). Esto implica que el error ante una entrada escalón es cero, confirmado por el cálculo del error en estado estacionario y la respuesta al escalón.

\begin{figure}[h!]
    \centering
    \includegraphics[width=0.3\textwidth]{/home/jorge/Documentos/FCEFyN/SC2/Laboret/Tarea1/capturas/figura6_1.png}
    \includegraphics[width=0.6\textwidth]{/home/jorge/Documentos/FCEFyN/SC2/Laboret/Tarea1/capturas/figura6_2.png}
    \caption{Sistema realimentado \(F(z)\) y \(e_{SS}\)(izq.). Respuesta al escalón (der.) }
\end{figure}

\subsection{Respuesta ante una rampa}
\begin{lstlisting}[language=Matlab]
t = 0:Tm:100*Tm;
lsim(F, t, t)
\end{lstlisting}

La salida sigue la rampa con un error constante, dado que el sistema puede seguir señales de tipo rampa pero con error proporcional a \(1/K_v\).

\begin{figure}[h!]
    \centering
    \includegraphics[width=0.7\textwidth]{/home/jorge/Documentos/FCEFyN/SC2/Laboret/Tarea1/capturas/figura7.png}
    \caption{Respuesta ante entrada tipo rampa}
\end{figure}

\newpage

\subsection{Lugar de raíces de los sistemas continuo y discreto}
\begin{lstlisting}[language=Matlab]
rlocus(G)
rlocus(Gd)
\end{lstlisting}

El sistema continuo es estable para toda \(K > 0\). El sistema discreto \(G_d\) es estable para \(K < 5.41\).

\begin{figure}[h!]
    \centering
    \includegraphics[width=0.7\textwidth]{/home/jorge/Documentos/FCEFyN/SC2/Laboret/Tarea1/capturas/figura8_1.png}
    \includegraphics[width=0.7\textwidth]{/home/jorge/Documentos/FCEFyN/SC2/Laboret/Tarea1/capturas/figura8_2.png}
    \caption{Lugar de raíces: continuo \(G\) (arr.) y discreto \(G_d\) (aba.)}
\end{figure}

\subsection{Efecto del aumento del periodo de muestreo en la estabilidad}
\begin{lstlisting}[language=Matlab]
rlocus(Gd1)
\end{lstlisting}

Al aumentar el tiempo de muestreo, la ganancia crítica de estabilidad disminuye. El sistema se vuelve inestable con menor ganancia: para \(K < 0.934\) sigue siendo estable.

\begin{figure}[h!]
    \centering
    \includegraphics[width=0.7\textwidth]{/home/jorge/Documentos/FCEFyN/SC2/Laboret/Tarea1/capturas/figura9.png}
    \caption{Lugar de raíces para sistema muestreado con mayor \(T_m\)}
\end{figure}

\section*{Conclusión}

En este trabajo se analizaron las diferencias entre un sistema continuo y su versión discreta obtenida con retención de orden cero. Se comprobó cómo el proceso de muestreo afecta la ubicación de los polos, la estabilidad y la respuesta del sistema.

Se vio que aumentar el tiempo de muestreo reduce la estabilidad del sistema discreto, y que la ganancia crítica baja en consecuencia. También se observó que el sistema discreto es de tipo 1, lo que garantiza error cero ante un escalón y error constante ante una rampa.

Por último, con el lugar de raíces se visualizó cómo cambia la estabilidad al variar la ganancia y el tiempo de muestreo, lo cual es fundamental para el diseño de controladores digitales.
\end{document}

