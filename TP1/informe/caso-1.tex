\documentclass{article}
\usepackage[spanish]{babel}
\usepackage[utf8]{inputenc}
\usepackage[T1]{fontenc}
\usepackage{amsmath, amssymb}  % para fórmulas matemáticas
\usepackage{siunitx}           % para unidades (opcional)
\usepackage{graphicx}          % si vas a incluir imágenes
\usepackage{hyperref}          % para hipervínculos
\usepackage{geometry}          % para márgenes
\begin{document}

\section{Item 1}
\subsection{Modelado de un sistema RLC en variables de estado}

Se parte de un circuito serie RLC como el mostrado en la figura, con una entrada de tensión $v_e(t)$ y como salida se toma la tensión en la resistencia, $v_R(t)$.

Para modelar este sistema en variables de estado, hay que identificar qué variables almacenan energía. En este caso, eso lo hacen el inductor y el capacitor, así que las variables de estado naturales a elegir son:

\begin{itemize}
    \item $x_1(t) = i(t)$, corriente en el inductor
    \item $x_2(t) = v_c(t)$, tensión sobre el capacitor
\end{itemize}

Estas variables son suficientes para describir el estado dinámico del circuito en cualquier instante de tiempo.

Aplicando la ley de Kirchhoff de tensiones al lazo del circuito, se tiene:
\[
v_e(t) = v_R(t) + v_L(t) + v_C(t)
\]

Y usando las relaciones constitutivas de los componentes:
\[
v_R(t) = Ri(t), \quad v_L(t) = L\frac{di(t)}{dt}, \quad i(t) = C\frac{dv_c(t)}{dt}
\]

Entonces, reemplazando:
\[
v_e(t) = R i(t) + L \frac{di(t)}{dt} + v_c(t)
\]
Y además:
\[
\frac{dv_c(t)}{dt} = \frac{i(t)}{C}
\]

Con esto, podemos armar el sistema de ecuaciones diferenciales y dejarlo en forma matricial-vectorial como:

\[
\dot{x}(t) = A x(t) + B u(t)
\]

Donde:
\[
x(t) =
\begin{bmatrix}
x_1(t) \\
x_2(t)
\end{bmatrix}
=
\begin{bmatrix}
i(t) \\
v_c(t)
\end{bmatrix},
\quad
u(t) = v_e(t)
\]

\[
A = 
\begin{bmatrix}
-\frac{R}{L} & -\frac{1}{L} \\
\frac{1}{C} & 0
\end{bmatrix}, \quad
B =
\begin{bmatrix}
\frac{1}{L} \\
0
\end{bmatrix}
\]

Además, si se quiere obtener la salida $y(t) = v_R(t) = R \cdot i(t)$, se define:
\[
y(t) = C x(t), \quad
C = 
\begin{bmatrix}
R & 0
\end{bmatrix}
\]

\subsection*{Parámetros y Simulación}

Se asignaron los siguientes valores:

\[
R = 220 \, \Omega, \quad L = 500 \, \text{mH}, \quad C = 2{,}2 \, \mu\text{F}
\]

La entrada $v_e(t)$ se define como una señal escalón que cambia de signo cada $10$ ms, alternando entre $+12$ V y $-12$ V. Las condiciones iniciales del sistema se consideraron nulas.

Antes de simular, se analizó la dinámica del sistema. Para eso se calcularon los polos a partir de la matriz $A$, o de forma equivalente, con la función de transferencia obtenida por Laplace. Esto permite ver si el sistema es subamortiguado, sobreamortiguado o críticamente amortiguado.

Con los polos, se determinó el tiempo de integración necesario. Si el sistema es oscilatorio, se eligió el paso como $h \approx \frac{1}{20 \cdot \omega_d}$, y si no, se basó en la dinámica más rápida. En este caso, se utilizó integración por Euler para simular la evolución de las variables de estado en el tiempo.

El tiempo total de simulación fue de $50$ ms, suficiente para observar varios ciclos del cambio de la entrada.

\subsection*{Conclusión}

Este trabajo permitió entender cómo, a partir del concepto de variables de estado, se puede modelar y simular un sistema físico real como el circuito RLC. Elegimos como variables de estado a las que representan el almacenamiento de energía (corriente en el inductor y tensión en el capacitor), con lo cual armamos las ecuaciones diferenciales que definen el comportamiento dinámico del sistema.

Después de armar la representación en el espacio de estados, usamos la matriz del sistema $A$ para analizar la dinámica interna, ver si es oscilatoria o no, y con eso elegir un paso de integración adecuado. En este caso, aplicamos el método de Euler para integrar numéricamente las ecuaciones del sistema.

De la simulación se puede observar cómo el sistema responde ante una entrada que cambia de $+12$ a $-12$ V cada 10 ms. El comportamiento de la corriente y la tensión en el capacitor refleja la dinámica esperada del sistema RLC: oscilaciones amortiguadas, con una respuesta que intenta seguir los cambios bruscos de la entrada pero lo hace de forma suave debido a la presencia del inductor y el capacitor, que actúan como filtros.

Este análisis es clave para entender el comportamiento temporal de sistemas eléctricos y para poder predecir su respuesta ante distintos tipos de entrada.

\section{Item 2}
\section*{Identificación de parámetros de un sistema RLC a partir de la respuesta al escalón}

\subsection*{Planteo del problema}
Se dispone de la respuesta medida de un circuito RLC serie frente a una excitación tipo escalón. El objetivo es estimar los valores de $R$, $L$ y $C$ del sistema a partir de los datos, usando el método de la respuesta al escalón y comparando los resultados con el modelo teórico.

\subsection*{Modelo del sistema}
El circuito RLC serie responde a la siguiente ecuación diferencial:

\begin{equation}
V_e(t) = L \frac{di(t)}{dt} + R i(t) + V_C(t)
\end{equation}

Dado que la corriente en un capacitor se relaciona con la tensión por:

\begin{equation}
i(t) = C \frac{dV_C(t)}{dt}
\end{equation}

Podemos reescribir la ecuación de entrada en función de la tensión en el capacitor:

\begin{equation}
V_e(t) = LC \frac{d^2V_C(t)}{dt^2} + RC \frac{dV_C(t)}{dt} + V_C(t)
\end{equation}

Aplicando transformada de Laplace (con condiciones iniciales nulas), se obtiene la función de transferencia teórica:

\begin{equation}
\frac{V_C(s)}{V_e(s)} = \frac{1}{LC s^2 + RC s + 1}
\end{equation}

\subsection*{Análisis inicial de la respuesta}
A partir del análisis gráfico de la respuesta medida del sistema se observó:

\begin{itemize}
    \item La salida $V_C(t)$ no presenta oscilaciones, por lo que se descartan polos complejos conjugados.
    \item La forma de la curva sugiere un sistema de segundo orden con \textbf{polos reales distintos}.
    \item Se observó una señal de entrada escalón con retardo. Para evitar complejidades, se seleccionó el primer escalón positivo sin retardo aparente.
\end{itemize}

\subsection*{Identificación por método de 3 puntos (polo-polo-cero)}
Se aplicó el método descripto para sistemas de segundo orden con polos reales distintos, considerando una función de transferencia del tipo:

\begin{equation}
G(s) = \frac{K(T_3 s + 1)}{(T_1 s + 1)(T_2 s + 1)}
\end{equation}

Donde:
\begin{itemize}
    \item $K$: ganancia de estado estacionario.
    \item $T_1, T_2$: constantes de tiempo asociadas a los polos reales.
    \item $T_3$: constante de tiempo asociada al cero (posible).
\end{itemize}

Se seleccionaron tres puntos de la salida equiespaciados temporalmente: $t_1$, $2t_1$, y $3t_1$, extrayendo los valores de $V_C$ correspondientes: $y_1$, $y_2$, $y_3$.

A partir de los valores normalizados:

\begin{align}
k_1 &= \frac{y_1 / A}{K} - 1 \\
k_2 &= \frac{y_2 / A}{K} - 1 \\
k_3 &= \frac{y_3 / A}{K} - 1
\end{align}

Se resolvió el siguiente sistema no lineal para obtener los valores de $\alpha_1 = e^{-t_1/T_1}$ y $\alpha_2 = e^{-t_1/T_2}$:

\begin{equation}
\beta = \frac{k_1 + \alpha_2}{\alpha_1 - \alpha_2}
\end{equation}

Luego, se despejan las constantes de tiempo:

\begin{align}
T_1 &= -\frac{t_1}{\ln \alpha_1} \\
T_2 &= -\frac{t_1}{\ln \alpha_2} \\
T_3 &= \beta(T_1 - T_2) + T_1
\end{align}

Con estos valores se construyó la función de transferencia estimada $\hat{G}(s)$.

\subsection*{Estimación de parámetros físicos del sistema}
La función de transferencia teórica no contiene el cero, por lo que se planteó:

\begin{equation}
G_{teo}(s) = \frac{1}{LC s^2 + RC s + 1}
\end{equation}

Para obtener los parámetros físicos, se estimó primero el valor de $C$ a partir de la derivada hacia atrás de $V_C(t)$:

\begin{equation}
C(t) = \frac{i(t)}{dV_C(t)/dt}
\end{equation}

Donde la derivada se calculó por diferencias finitas hacia atrás:

\begin{equation}
\frac{dV_C(t)}{dt} \approx \frac{V_C(t) - V_C(t - \Delta t)}{\Delta t}
\end{equation}

Una vez conocido $C$, se utilizaron los coeficientes del denominador estimado para calcular:

\begin{align}
R &= \text{coeficiente de } s \div C \\
L &= \text{coeficiente de } s^2 \div C
\end{align}

\subsection*{Comparación de modelos}
Se graficaron en conjunto las siguientes curvas:
\begin{itemize}
    \item Respuesta medida del sistema.
    \item Respuesta del modelo estimado $\hat{G}(s)$.
    \item Respuesta del modelo teórico $G_{teo}(s)$ con los valores de $R$, $L$ y $C$ obtenidos.
    \item Respuesta del modelo teórico \textbf{con el cero} (modelo ampliado).
\end{itemize}

\subsection*{Observaciones}
\begin{itemize}
    \item La función estimada presenta un cero (en el numerador) que no se encuentra en el modelo teórico clásico de un RLC serie, pero que surge del método de identificación basado en la salida.
    \item Este cero resulta ser muy cercano al origen ($T_3 \ll 1$), por lo cual puede despreciarse para simplificar la comparación con el modelo físico.
    \item Se observa una ligera diferencia entre la curva teórica sin cero y la respuesta real en la zona de subida. Esta diferencia se reduce significativamente al incluir el cero estimado.
\end{itemize}

\subsection*{Conclusiones}
\begin{itemize}
    \item Se logró estimar adecuadamente una función de transferencia a partir de datos experimentales, utilizando el método de los tres puntos para sistemas con polos reales distintos.
    \item A partir del modelo estimado se obtuvieron valores razonables de $R$, $L$ y $C$ que reproducen con buena precisión la dinámica del sistema.
    \item La inclusión del cero en el modelo mejora el ajuste con la señal medida, lo que demuestra la utilidad del modelo de identificación, aún cuando no coincida estructuralmente con el circuito real.
    \item El método demuestra ser útil para sistemas de segundo orden sin oscilaciones, donde se pueda asumir un modelo de polos reales distintos.
\end{itemize}

\section{Item 3}
\subsection*{Verificación del modelo a partir del segundo escalón negativo}

Para validar los parámetros estimados en los ítems anteriores, se utilizó el segundo escalón negativo de la señal de entrada, el cual ocurre aproximadamente a los $0.05\,$s. A partir de este punto, se extrajeron las señales medidas de corriente $i(t)$, tensión de entrada $v_i(t)$ y tensión en el capacitor $v_c(t)$.

Se normalizó el eje temporal para que el instante inicial coincida con el comienzo del escalón (es decir, $t=0$). Luego, se estimó la corriente utilizando el valor de capacitancia $C$ previamente calculado, a partir de la ecuación:

\[
i_{\text{estimada}}(t) = C \cdot \frac{d v_c(t)}{dt}
\]

Para la derivada temporal, se aplicó el método de diferencias hacia atrás sobre la señal $v_c(t)$, considerando el paso de tiempo constante $\Delta t$ obtenido de los datos medidos.

A continuación, se presenta la comparación entre la corriente estimada y la medida:

%\begin{figure}[h!]
  %  \centering
   % \includegraphics[width=0.9\textwidth]{figuras/comparacion_corriente_verificacion.png}
    %\caption{Comparación entre corriente medida y corriente estimada a partir de la derivada de $v_c(t)$, utilizando la capacitancia estimada en el ítem anterior.}
    %\label{fig:verificacion_corriente}
%\end{figure}

\subsubsection*{Conclusiones}

La superposición entre la corriente medida y la corriente estimada es razonablemente buena, especialmente en la forma general de la respuesta y el pico inicial. Esto valida el valor de la capacitancia $C$ previamente estimado, así como la hipótesis de que el modelo obtenido describe adecuadamente el comportamiento dinámico del circuito RLC en torno a transiciones de entrada.

Sin embargo, se observan pequeñas diferencias en la amplitud y en el decaimiento, lo cual puede atribuirse a:
\begin{itemize}
    \item Aproximaciones numéricas en el cálculo de la derivada.
    \item Limitaciones de precisión en los datos medidos.
    \item Efectos no modelados como la resistencia parásita del capacitor o pequeñas no linealidades.
\end{itemize}

En conjunto, este análisis respalda la validez del modelo obtenido a partir del primer escalón positivo, demostrando su capacidad para reproducir correctamente el comportamiento del sistema ante nuevas excitaciones.


\end{document}
