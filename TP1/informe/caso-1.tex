\documentclass{article}
\usepackage[spanish]{babel}
\usepackage[utf8]{inputenc}
\usepackage[T1]{fontenc}
\usepackage{amsmath, amssymb}  % para fórmulas matemáticas
\usepackage{siunitx}           % para unidades (opcional)
\usepackage{graphicx}          % si vas a incluir imágenes
\usepackage{hyperref}          % para hipervínculos
\usepackage{geometry}          % para márgenes
\begin{document}

\section{Modelado de un sistema RLC en variables de estado}

Se parte de un circuito serie RLC como el mostrado en la figura, con una entrada de tensión $v_e(t)$ y como salida se toma la tensión en la resistencia, $v_R(t)$.

Para modelar este sistema en variables de estado, hay que identificar qué variables almacenan energía. En este caso, eso lo hacen el inductor y el capacitor, así que las variables de estado naturales a elegir son:

\begin{itemize}
    \item $x_1(t) = i(t)$, corriente en el inductor
    \item $x_2(t) = v_c(t)$, tensión sobre el capacitor
\end{itemize}

Estas variables son suficientes para describir el estado dinámico del circuito en cualquier instante de tiempo.

Aplicando la ley de Kirchhoff de tensiones al lazo del circuito, se tiene:
\[
v_e(t) = v_R(t) + v_L(t) + v_C(t)
\]

Y usando las relaciones constitutivas de los componentes:
\[
v_R(t) = Ri(t), \quad v_L(t) = L\frac{di(t)}{dt}, \quad i(t) = C\frac{dv_c(t)}{dt}
\]

Entonces, reemplazando:
\[
v_e(t) = R i(t) + L \frac{di(t)}{dt} + v_c(t)
\]
Y además:
\[
\frac{dv_c(t)}{dt} = \frac{i(t)}{C}
\]

Con esto, podemos armar el sistema de ecuaciones diferenciales y dejarlo en forma matricial-vectorial como:

\[
\dot{x}(t) = A x(t) + B u(t)
\]

Donde:
\[
x(t) =
\begin{bmatrix}
x_1(t) \\
x_2(t)
\end{bmatrix}
=
\begin{bmatrix}
i(t) \\
v_c(t)
\end{bmatrix},
\quad
u(t) = v_e(t)
\]

\[
A = 
\begin{bmatrix}
-\frac{R}{L} & -\frac{1}{L} \\
\frac{1}{C} & 0
\end{bmatrix}, \quad
B =
\begin{bmatrix}
\frac{1}{L} \\
0
\end{bmatrix}
\]

Además, si se quiere obtener la salida $y(t) = v_R(t) = R \cdot i(t)$, se define:
\[
y(t) = C x(t), \quad
C = 
\begin{bmatrix}
R & 0
\end{bmatrix}
\]

\subsection*{Parámetros y Simulación}

Se asignaron los siguientes valores:

\[
R = 220 \, \Omega, \quad L = 500 \, \text{mH}, \quad C = 2{,}2 \, \mu\text{F}
\]

La entrada $v_e(t)$ se define como una señal escalón que cambia de signo cada $10$ ms, alternando entre $+12$ V y $-12$ V. Las condiciones iniciales del sistema se consideraron nulas.

Antes de simular, se analizó la dinámica del sistema. Para eso se calcularon los polos a partir de la matriz $A$, o de forma equivalente, con la función de transferencia obtenida por Laplace. Esto permite ver si el sistema es subamortiguado, sobreamortiguado o críticamente amortiguado.

Con los polos, se determinó el tiempo de integración necesario. Si el sistema es oscilatorio, se eligió el paso como $h \approx \frac{1}{20 \cdot \omega_d}$, y si no, se basó en la dinámica más rápida. En este caso, se utilizó integración por Euler para simular la evolución de las variables de estado en el tiempo.

El tiempo total de simulación fue de $50$ ms, suficiente para observar varios ciclos del cambio de la entrada.

\subsection*{Conclusión}

Este trabajo permitió entender cómo, a partir del concepto de variables de estado, se puede modelar y simular un sistema físico real como el circuito RLC. Elegimos como variables de estado a las que representan el almacenamiento de energía (corriente en el inductor y tensión en el capacitor), con lo cual armamos las ecuaciones diferenciales que definen el comportamiento dinámico del sistema.

Después de armar la representación en el espacio de estados, usamos la matriz del sistema $A$ para analizar la dinámica interna, ver si es oscilatoria o no, y con eso elegir un paso de integración adecuado. En este caso, aplicamos el método de Euler para integrar numéricamente las ecuaciones del sistema.

De la simulación se puede observar cómo el sistema responde ante una entrada que cambia de $+12$ a $-12$ V cada 10 ms. El comportamiento de la corriente y la tensión en el capacitor refleja la dinámica esperada del sistema RLC: oscilaciones amortiguadas, con una respuesta que intenta seguir los cambios bruscos de la entrada pero lo hace de forma suave debido a la presencia del inductor y el capacitor, que actúan como filtros.

Este análisis es clave para entender el comportamiento temporal de sistemas eléctricos y para poder predecir su respuesta ante distintos tipos de entrada.

\end{document}
