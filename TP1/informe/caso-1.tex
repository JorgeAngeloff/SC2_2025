%%Prximo a hacer
%FT estimadas y teroicas que me dieron y capturas

\documentclass{article}
\usepackage[spanish]{babel}
\usepackage[utf8]{inputenc}
\usepackage[T1]{fontenc}
\usepackage{amsmath, amssymb}  % para fórmulas matemáticas
\usepackage{siunitx}           % para unidades (opcional)
\usepackage{graphicx}          % si vas a incluir imágenes
\usepackage{hyperref}          % para hipervínculos
\usepackage{geometry}          % para márgenes

\usepackage{listings}

\begin{document}

\section{Caso de estudio 1: Sistema de dos variables de estado}
\subsection*{Ítem [1]: Modelado de un sistema RLC en variables de estado}

Se parte de un circuito serie RLC como el mostrado en la figura, con una entrada de tensión $v_e(t)$ y como salida se toma la tensión en la resistencia, $v_R(t)$.

Para modelar este sistema en variables de estado, hay que identificar qué variables almacenan energía. En este caso, eso lo hacen el inductor y el capacitor, así que las variables de estado naturales a elegir son:

\begin{itemize}
    \item $x_1(t) = i(t)$, corriente en el inductor
    \item $x_2(t) = v_c(t)$, tensión sobre el capacitor
\end{itemize}

Estas variables son suficientes para describir el estado dinámico del circuito en cualquier instante de tiempo.

Aplicando la ley de Kirchhoff de tensiones al lazo del circuito, se tiene:
\[
v_e(t) = v_R(t) + v_L(t) + v_C(t)
\]

Y usando las relaciones constitutivas de los componentes:
\[
v_R(t) = Ri(t), \quad v_L(t) = L\frac{di(t)}{dt}, \quad i(t) = C\frac{dv_c(t)}{dt}
\]

Entonces, reemplazando:
\[
v_e(t) = R i(t) + L \frac{di(t)}{dt} + v_c(t)
\]
Y además:
\[
\frac{dv_c(t)}{dt} = \frac{i(t)}{C}
\]

Con esto, podemos armar el sistema de ecuaciones diferenciales y dejarlo en forma matricial-vectorial como:

\[
\dot{x}(t) = A x(t) + B u(t)
\]

Donde:
\[
x(t) =
\begin{bmatrix}
x_1(t) \\
x_2(t)
\end{bmatrix}
=
\begin{bmatrix}
i(t) \\
v_c(t)
\end{bmatrix},
\quad
u(t) = v_e(t)
\]

\[
A = 
\begin{bmatrix}
-\frac{R}{L} & -\frac{1}{L} \\
\frac{1}{C} & 0
\end{bmatrix}, \quad
B =
\begin{bmatrix}
\frac{1}{L} \\
0
\end{bmatrix}
\]

La salida se define:
\[
y(t) = C x(t) \]
Donde: 
\[y(t) = v_R(t) = R \cdot i(t)\] 
 
\[
C = 
\begin{bmatrix}
R & 0
\end{bmatrix}
\]

\subsection*{Parámetros y Simulación}

Se asignaron los siguientes valores:

\[
R = 220 \, \Omega, \quad L = 500 \, \text{mH}, \quad C = 2{,}2 \, \mu\text{F}
\]

La entrada $v_e(t)$ se define como una señal escalón que cambia de signo cada $10$ ms, alternando entre $+12$ V y $-12$ V. Las condiciones iniciales del sistema se consideraron nulas.

Antes de simular, se analizó la dinámica del sistema. Para eso se calcularon los polos a partir de la matriz $A$. Calculando el determinante \( \det|sI - A| \) obtenemos el polinomio caracteristico del sistema, para luego obtener los polos.

Resulta: 

$$S_{1,2}=-220 \, \pm J \, 927.74 $$

Con estos polos podemos deducir que se trata de un sistema subamortiguado. Al ser un sistema con polos complejos conjugados, se calculo el tiempo de integracion en base a la frecuencia natural amortiguada,  $h \approx \frac{2\pi}{100 \cdot \omega_d}$. Despejando $\omega_d$ de los polos obtenidos. 

\[w_d=927.74 \, \frac{rad}{seg}\] 
\[ h=\frac{2\pi}{100 \cdot \omega_d} =67.7\,\mu seg\]


En este caso, se utilizó integración por Euler para simular la evolución de las variables de estado en el tiempo.
\begin{equation*}
    x(k+1) = x(k) + h \cdot \dot{x}(k)
\end{equation*}
El tiempo total de simulación fue de $50$ ms, suficiente para observar varios ciclos del cambio de la entrada.

\begin{figure}[h!t]
      \centering
        \includegraphics[width=0.8\textwidth]{/home/jorge/Documentos/FCEFyN/SC2/Pucheta/TP1/Capturas/item1}
          \caption{Simulacion del modelo.}
            \label{fig:etiqueta}
        \end{figure} 

\subsection*{Conclusión}

Este punto permitió entender cómo, a partir del concepto de variables de estado, se puede modelar y simular un sistema físico real como el circuito RLC. 

De la simulación se puede observar cómo el sistema responde ante una entrada que cambia de $+12$ a $-12$ V cada 10 ms. El comportamiento de la corriente y la tensión en el capacitor refleja la dinámica esperada del sistema RLC: oscilaciones amortiguadas, con una respuesta que intenta seguir los cambios bruscos de la entrada pero lo hace de forma suave debido a la presencia del inductor y el capacitor, que actúan como filtros.



\subsection*{ Ítem [2] Identificación de parámetros de un sistema RLC a partir de la respuesta al escalón}

Se dispone de la respuesta medida de un circuito RLC serie frente a una excitación tipo escalón que empieza en $t=0.01 \, seg$ y cada $\Delta t=0.05 \, seg$ la aplitud del escalon cambia entre $V_e=12 \, V$ y $V_e=-12 \, V $. El objetivo es estimar los valores de $R$, $L$ y $C$ del sistema a partir de los datos ($v_e(t)$, $v_c(t)$ e $i(t)$), usando el método de la respuesta al escalón y comparando los resultados con el modelo teórico.

\subsection*{Modelo del sistema RLC serie teorico con $v_c$ como salida.}
El circuito RLC serie responde a la siguiente ecuación diferencial:

\begin{equation}
V_e(t) = L \frac{di(t)}{dt} + R i(t) + V_C(t)
\end{equation}

Dado que la corriente en un capacitor se relaciona con la tensión por:

\begin{equation}
i(t) = C \frac{dV_C(t)}{dt}
\end{equation}

Podemos reescribir la ecuación de entrada en función de la tensión en el capacitor:

\begin{equation}
V_e(t) = LC \frac{d^2V_C(t)}{dt^2} + RC \frac{dV_C(t)}{dt} + V_C(t)
\end{equation}

Aplicando transformada de Laplace (con condiciones iniciales nulas), se obtiene la función de transferencia teórica:

\begin{equation}
    \frac{V_C(s)}{V_e(s)} =G_{teo}(s)= \frac{1}{LC s^2 + RC s + 1}
\end{equation}


\subsection*{Identificación por método de Chen para polos reales distintos}
Debido a la forma de la funcion de transferencia del circuito RLC, con dos polos y un cero, se aplicó el método descripto para sistemas de segundo orden con polos reales distintos, considerando una función de transferencia del tipo:

\begin{equation}
G(s) = \frac{K(T_3 s + 1)}{(T_1 s + 1)(T_2 s + 1)}
\end{equation}

Donde:
\begin{itemize}
    \item $K$: ganancia de estado estacionario.
    \item $T_1, T_2$: constantes de tiempo asociadas a los polos reales.
    \item $T_3$: constante de tiempo asociada al cero.
\end{itemize}


Se seleccionaron tres puntos de la salida equiespaciados temporalmente: $t_1$, $2t_1$, y $3t_1$, extrayendo los valores de $V_C$ correspondientes: $y_1$, $y_2$, $y_3$.

A partir de los valores normalizados:

\begin{align}
k_1 &= \frac{y_1 / A}{K} - 1 \\
k_2 &= \frac{y_2 / A}{K} - 1 \\
k_3 &= \frac{y_3 / A}{K} - 1
\end{align}

Se resolvió el siguiente sistema no lineal para obtener los valores de $\alpha_1 = e^{-t_1/T_1}$ y $\alpha_2 = e^{-t_1/T_2}$:

\begin{equation}
\beta = \frac{k_1 + \alpha_2}{\alpha_1 - \alpha_2}
\end{equation}

Luego, se despejan las constantes de tiempo:

\begin{align}
T_1 &= -\frac{t_1}{\ln \alpha_1} \\
T_2 &= -\frac{t_1}{\ln \alpha_2} \\
T_3 &= \beta(T_1 - T_2) + T_1
\end{align}

Con estos valores se construyó la función de transferencia estimada $\hat{G}(s)$.
\begin{equation}
    \hat{G}(s) = \frac{K(T_3 s + 1)}{(T_1 s + 1)(T_2 s + 1)}
\end{equation}

\begin{figure}[!ht]
      \centering
        \includegraphics[width=0.8\textwidth]{/home/jorge/Documentos/FCEFyN/SC2/Pucheta/TP1/Capturas/item2_estimada}
          \caption{Comparacion de la respuesta al escalon de la funcion de transferencia real vs la estimada}
            \label{fig:etiqueta2}
        \end{figure} 


\subsection*{Estimación de parámetros físicos del sistema}
La función de transferencia teórica no contiene el cero, como ya vimos nos queda:

\begin{equation}
G_{teo}(s) = \frac{1}{LC s^2 + RC s + 1}
\end{equation}

Para obtener los parámetros físicos, se estimó primero el valor de $C$ a partir de la derivada hacia atrás de $V_C(t)$:

\begin{equation}
C(t) = \frac{i(t)}{dV_C(t)/dt}
\end{equation}

Donde la derivada se calculó por diferencias finitas hacia atrás:

\begin{equation}
\frac{dV_C(t)}{dt} \approx \frac{V_C(t) - V_C(t - \Delta t)}{\Delta t}
\end{equation}

Una vez conocido $C$, se utilizaron los coeficientes del denominador estimado para calcular:

\begin{align}
R &= \text{coeficiente de } s \div C \\
L &= \text{coeficiente de } s^2 \div C
\end{align}

Obtengo:
\begin{center}
\begin{tabular}{lc}
\hline
Componente & Valor \\
\hline
\(R\) & \(282.3\ \Omega\) \\
\(L\) & \(29.4 \ \mathrm{mH}\) \\
\(C\) & \(2.177 \ \mathrm{\mu F}\) \\
\hline
\end{tabular}
\end{center}
\subsection*{Comparación de modelos}
Se graficaron en conjunto las siguientes curvas:
\begin{itemize}
    \item Respuesta medida del sistema.
    \item Respuesta del modelo estimado $\hat{G}(s)$.
    \item Respuesta del modelo teórico $G_{teo}(s)$ con los valores de $R$, $L$ y $C$ obtenidos.
    \item Respuesta del modelo teórico \textbf{con el cero} obtenido en el modelo estimado.
\end{itemize}

\begin{figure}[ht!]
      \centering
        \includegraphics[width=0.8\textwidth]{/home/jorge/Documentos/FCEFyN/SC2/Pucheta/TP1/Capturas/item2_comparacion}
          \caption{Comparacion de los resultados obtenidos.}
            \label{fig:etiqueta3}
        \end{figure} 

\subsection*{Observaciones}
\begin{itemize}
    \item La función estimada presenta un cero (en el numerador) que no se encuentra en el modelo teórico clásico de un RLC serie, pero que surge del método de identificación basado en la salida.
    \item Este cero resulta ser muy cercano al origen ($T_3 \ll 1$), por lo cual puede despreciarse para simplificar la comparación con el modelo físico.
    \item Se observa una ligera diferencia entre la curva teórica sin cero y la respuesta real en la zona de subida. Esta diferencia se reduce significativamente al incluir el cero estimado.
\end{itemize}

\subsection*{Conclusiones}
\begin{itemize}
    \item Se logró estimar adecuadamente una función de transferencia a partir de datos experimentales, utilizando el método Chen de los tres puntos para sistemas con polos reales distintos.
    \item A partir del modelo estimado se obtuvieron valores razonables de $R$, $L$ y $C$ <que reproducen con buena precisión la dinámica del sistema.
    \item La inclusión del cero en el modelo mejora el ajuste con la señal medida, lo que demuestra la utilidad del modelo de identificación, aún cuando no coincida estructuralmente con el circuito real.
    \item El método demuestra ser útil para sistemas de segundo orden sin oscilaciones, donde se pueda asumir un modelo de polos reales distintos.
\end{itemize}

\subsection{Item 3: Verificación del modelo a partir del segundo escalón negativo y la corriente}

Para validar los parámetros estimados en el ítem 2, se utilizó el segundo escalón negativo de la señal de entrada, el cual ocurre aproximadamente a los $0.05\,$s. A partir de este punto, se extrajeron las señales medidas de corriente $i(t)$, tensión de entrada $v_i(t)$ y tensión en el capacitor $v_c(t)$.

Se normalizó el eje temporal para que el instante inicial coincida con el comienzo del escalón (es decir, $t=0$). Luego, se estimó la corriente utilizando el valor de capacitancia $C$ previamente calculado, a partir de la ecuación:

\[
i_{\text{estimada}}(t) = C \cdot \frac{d v_c(t)}{dt}
\]

Para la derivada temporal, se aplicó el método de diferencias hacia atrás sobre la señal $v_c(t)$, considerando el paso de tiempo constante $\Delta t$ obtenido de los datos medidos.

A continuación, se presenta la comparación entre la corriente estimada y la medida:

\begin{figure}[ht!]
      \centering
        \includegraphics[width=0.8\textwidth]{/home/jorge/Documentos/FCEFyN/SC2/Pucheta/TP1/Capturas/item3}
          \caption{Verificacion de los componentes con la corriente.}
            \label{fig:etiqueta4}
        \end{figure} 

\subsubsection*{Conclusiones}

La superposición entre la corriente medida y la corriente estimada es razonablemente buena, especialmente en la forma general de la respuesta y el pico inicial. Esto valida el valor de la capacitancia $C$ estimada, así como el modelo obtenido y describe adecuadamente el comportamiento dinámico del circuito RLC en torno a transiciones de entrada. Esto podria ser util para dimensionar los componentes reales por ejemplo.

Este análisis respalda la validez del modelo obtenido a partir del primer escalón positivo, demostrando su capacidad para reproducir correctamente el comportamiento del sistema ante nuevas excitaciones.

\section{Caso de estudio 2. Sistema de tres variables de estado.}
\subsection*{Ítem [4] - Torque máximo soportado por el motor con controlador PID}

Determinar el torque de carga máximo que puede soportar el motor modelado, alimentado como maximo con \SI{12}{\volt},  evaluando la respuesta del sistema controlado con un PID discreto. Se busca identificar el valor de $T_L$ a partir del cual el motor no puede mantener su velocidad de referencia o directamente se detenga.

Se utilizó un modelo del motor a partir del modelado de estados implementado mediante integración por el método de Euler con paso temporal $\Delta t = \SI{1e-7}{\second}$. El sistema es regulado por un controlador PID con parámetros:
\begin{center}
\begin{tabular}{l}
  \hline
  $K_p = 6$   \\
  $K_i = 920$ \\
  $K_d = 0$   \\
  \hline
\end{tabular}
\end{center}

El torque de carga $T_L$ se incrementa linealmente en cada paso temporal con un valor de $\Delta T_L = \SI{1e-12}{Nm}$ hasta un tiempo total de simulación de $25 \, mseg$. Se registran la velocidad angular $\omega(t)$ y el valor de $T_L(t)$ para identificar el instante en el que la velocidad angular comienza a decaer y no puede mantenerse próxima a la referencia impuesta de \SI{20}{rad/s} o cruza por cero.


\subsubsection*{Resultados}

\begin{figure}[ht!]
      \centering
        \includegraphics[width=0.8\textwidth]{/home/jorge/Documentos/FCEFyN/SC2/Pucheta/TP1/Capturas/item4}
          \caption{Carga maxima para detener el motor.}
            \label{fig:etiqueta5}
        \end{figure} 

La figura muestra cómo la velocidad angular $\omega(t)$ decrece a medida que el torque de carga incrementa, hasta llegar a un punto donde el motor no puede sostener la velocidad de referencia. Observando el cruce por cero de la velocidad, se puede estimar que el torque máximo soportado bajo estas condiciones de control es:

\[
T_{L, \text{máx}} \approx \SI{2.2e-5}{Nm}
\]


\subsubsection*{Observaciones}
\begin{itemize}
    \item Es importante considerar este valor como una referencia para dimensionar adecuadamente el hardware de control y potencia.
\end{itemize}



\subsection{Ítem 5: Identificación de funciones de transferencia a partir de datos experimentales}

A partir de las curvas medidas del archivo \texttt{Curvas\_Medidas\_Motor\_2025.xls}, se buscó obtener un modelo que describa cómo se comporta el motor de corriente continua. La entrada fue un escalón de $2\ \text{V}$ y la salida analizada, en un principio, fue la velocidad angular $\omega(t)$. También se intentó ver cómo afecta el torque de carga $T_L$ como una perturbación.


Se aplicó el mismo metodo usado en el caso 1, Chen de los tres puntos que sirve para obtener funciones de transferencia de segundo orden partir de la respuesta a un escalón. El modelo general que se plantea es:

\begin{equation}
    G(s) = K \cdot \frac{T_3 s + 1}{(T_1 s + 1)(T_2 s + 1)}
\end{equation}

Para aplicar el método, se eligió una parte del experimento donde se ve un escalón claro, se reacomodó el eje del tiempo para que ese escalón empiece en $t = 0$, y se eligieron tres puntos de la curva que son clave para estimar los parámetros. Con eso, se calcularon las constantes de tiempo $T_1$, $T_2$, $T_3$ y la ganancia $K$ del sistema.

\subsection*{Estimación de la función de transferencia \texorpdfstring{$\omega/V_a$}{w/Va}}

A partir de la entrada escalón $V_a(t)$ y la salida velocidad angular $\omega(t)$, aplicando el método se obtuvo un modelo que representaba bien el comportamiento del motor. La forma fue:


\[
\frac{\omega(s)}{V_a(s)} = K \cdot \frac{T_3 s + 1}{(T_1 s + 1)(T_2 s + 1)}
\]

Cuando se comparó la respuesta del modelo con los datos reales, se vio que el ajuste era bueno. Sin embargo, desde el punto de vista físico, la función de transferencia que debería tener el motor (según el modelo teórico) es:

\begin{equation}
    \frac{\omega(s)}{V_a(s)} = \frac{K_m K_i}{L_a J_m s^2 + (L_a B_m + R_a J_m) s + (R_a B_m + K_m K_i)}
\end{equation}

Este modelo teórico no tiene ceros, así que el cero que aparece en la estimación por Chen no tiene sentido físico. Por eso, se decidió dejar sólo con el denominador de la estimación (que sí tiene sentido con la dinámica del sistema) y no tener en cuenta el numerador.

\subsection*{Dificultades para estimar \texorpdfstring{$\omega/T_L$}{w/TL}}

Se intentó también obtener un modelo para $\omega/T_L$, considerando el torque de carga como la entrada. Aunque se pudo ajustar una curva, no fue una estimación confiable. Esto se debe a que durante esa parte de los datos extraidos, la entrada $V_a$ seguía activa y, además, las condiciones iniciales no eran cero. Entonces, no se puede decir que la respuesta observada fue únicamente causada por el torque de carga, así que no se pudo identificar bien ese modelo.

\subsection*{Estimación de la función de transferencia \texorpdfstring{$i_a/V_a$}{ia/Va}}
Para tener una segunda ecuación y así poder obtener los parámetros del motor, se tomó ahora como salida la corriente de armadura $i_a(t)$, manteniendo $V_a(t)$ como entrada. En este caso, la función teórica sí tiene un cero real, y es:

\begin{equation}
    \frac{i_a(s)}{V_a(s)} = \frac{J_m s + B_m}{L_a J_m s^2 + (L_a B_m + R_a J_m) s + (R_a B_m + K_m K_i)}
\end{equation}

Como el denominador es el mismo que en $\omega/V_a$, y el numerador ahora sí tiene un cero real, se pudo usar la estimación completa obtenida con el método de Chen y comparar con el modelo teórico para identificar los parámetros físicos.

\subsection*{Parámetros identificados}
Partimos de calcular $R_a$ como la tensión dividida la corriente inicial del escalón:

$$R_a = \frac{V_a}{I_{a(pico)}} =2.415 \, \Omega$$ 

De la comparación entre los modelos estimados y teóricos, se llegó a los siguientes valores para los parámetros del modelo:

\begin{align*}
    R_a &= 2.415\ \Omega \\
    L_a &= 0.002511\ \text{H} \\
    J_m &= 0.02056\ \text{kg·m}^2 \\
    B_m &= 0.026\ \text{N·m·s} \\
    K_i &= 3.765\ \text{V·s/rad} \\
    K_m &= 0.2485\ \text{N·m/A}
\end{align*}

Con estos datos, se armaron las funciones de transferencia teóricas $\omega/V_a$ y $I_a/V_a$, y se las comparo con las curvas de los datos y las estimadas.

\subsection*{Verificación y resultados}

Se graficaron las respuestas de los modelos estimados y teóricos junto con las curvas reales. En general, se vio que ambos modelos representaban bien el comportamiento del motor, tanto para la velocidad angular como para la corriente de armadura.

\subsection*{Concluciones}
En resumen, se pudo encontrar un modelo dinámico confiable para el motor, usando datos experimentales y un método de identificación temporal. Además, se lograron obtener los parámetros físicos del sistema, lo que es muy útil para trabajos futuros de simulación o diseño de controladores.


\section{Ítem 6: Control PID discreto del ángulo del motor}

\subsection*{Consigna}
Implementar un PID en tiempo discreto para que el ángulo del motor permanezca en una referencia de 1~rad, sometido al torque descripto en la Fig. 1-3. \emph{Tip: partir de $K_P = 0.1$, $K_I = 0.01$, $K_D = 5$}.

\subsection*{Implementación y desarrollo}

El controlador PID se implementó en forma discreta utilizando la ecuación general de diferencias basada en el método trapezoidal:

\[
u[k] = u[k{-}1] + A_1 e[k] + B_1 e[k{-}1] + C_1 e[k{-}2]
\]

donde:

\begin{align*}
A_1 &= \frac{2K_P T_s + K_I T_s^2 + 2K_D}{2T_s} \\
B_1 &= \frac{-2K_P T_s + K_I T_s^2 - 4K_D}{2T_s} \\
C_1 &= \frac{K_D}{T_s}
\end{align*}

El sistema se simuló utilizando el modelo del motor obtenido previamente por identificación, con la entrada de control $u$ limitada entre $[-12\ \text{V},\ 12\ \text{V}]$ para reflejar saturación del actuador.

Durante la simulación se observó que, con el paso de integración original $T_s = 1\text{ ms}$, el error acumulado era significativo debido al comportamiento del término integral. Para mejorar la precisión numérica, se incrementó el paso a $T_s = 10\text{ ms}$, lo que redujo la acumulación de error y estabilizó la respuesta.

\subsection*{Código MATLAB utilizado}

\begin{verbatim}
clear;
X = -[0; 0; 0; 0];
ii = 0;
t_etapa = 1e-2;      % Paso de integración ampliado
thetaRef = 1;        % Referencia de ángulo [rad]
tF = 10;             % Tiempo final de simulación
TL = 0;

% Constantes del PID ajustadas
Kp = 2000;
Ki = 5000;
Kd = 10;

Ts = t_etapa;
A1 = ((2*Kp*Ts) + (Ki*(Ts^2)) + (2*Kd)) / (2*Ts);
B1 = (-2*Kp*Ts + Ki*(Ts^2) - 4*Kd) / (2*Ts);
C1 = Kd / Ts;

e = zeros(tF / t_etapa, 1); 
u = 0;

for t = 0:t_etapa:tF
    ii = ii + 1; k = ii + 2;

    % Torque externo aplicado según consigna
    if(t >= 0.7010)
        TL = 0.12;
    end
    if(t >= 1.001)
        TL = 0;
    end

    X = modmotor_ang(t_etapa, X, u, TL);
    e(k) = thetaRef - X(3); % Error de posición angular
    u = u + A1*e(k) + B1*e(k-1) + C1*e(k-2); % PID discreto

    % Saturación del actuador
    u = max(min(u, 12), -12); 

    % Registro de variables
    x1(ii) = X(1); % Velocidad angular
    x2(ii) = X(2); % Wp
    x3(ii) = X(3); % Ángulo
    x4(ii) = X(4); % Corriente
    acc(ii) = u;
    carga(ii) = TL;
end

% Gráficos
t = 0:t_etapa:tF;
subplot(3,1,1); hold on;
plot(t,x3,'r'); title('Salida \theta'); xlabel('Tiempo [s]');
subplot(3,1,2); hold on;
plot(t,carga,'m'); title('Carga TL'); xlabel('Tiempo [s]');
subplot(3,1,3); hold on;
plot(t,acc,'b'); title('Entrada u_t'); xlabel('Tiempo [s]');
\end{verbatim}

\subsection*{Observaciones}

\begin{itemize}
    \item La respuesta del sistema con los valores iniciales sugeridos ($K_P = 0.1$, $K_I = 0.01$, $K_D = 5$) fue muy lenta, por lo que se aumentaron considerablemente las ganancias para lograr una regulación aceptable.
    \item El aumento del paso de integración ($T_s = 10$~ms) ayudó a estabilizar la simulación y evitar errores numéricos acumulados, especialmente en el término integral.
    \item Una oscilación pequeña en régimen permanente ($\pm 0.02$~rad) fue observada, producto de la ausencia de una componente derivativa en etapas anteriores. Se incorporó $K_D > 0$ para mejorar el amortiguamiento.
\end{itemize}

\subsection*{Conclusiones}

Se logró implementar un controlador PID discreto efectivo que regula el ángulo del motor a 1~rad pese a perturbaciones externas. Fue fundamental ajustar las ganancias a valores más altos que los sugeridos inicialmente y agregar la acción derivativa para evitar oscilaciones sostenidas en régimen. El paso de integración también tuvo un rol clave: si es muy pequeño, se acumulan errores numéricos; si es muy grande, se pierde precisión. Se encontró un equilibrio con $T_s = 10$~ms.

El modelo del motor utilizado corresponde al identificado previamente a partir de curvas experimentales, lo cual permitió simular de manera realista la dinámica del sistema.
\section{Ítem 6}

Se implementó un controlador PID en tiempo discreto para regular el ángulo \(\theta(t)\) de un motor de corriente continua hacia una referencia constante de \(1\,\text{rad}\), considerando un perfil de torque de carga variable \(T_L(t)\), tal como se describe en la Fig. 1-3.

El sistema fue simulado en MATLAB mediante un modelo dinámico del motor basado en sus ecuaciones diferenciales físicas. Las variables de estado consideradas fueron la velocidad angular \(\omega\), su derivada \(w_p\), el ángulo \(\theta\) y la corriente del inducido \(i_a\). Las ecuaciones se integraron utilizando el método de Euler explícito con paso constante \(h\). El modelo implementado fue:

\begin{align*}
    \dot{\omega} &= w_p \\
    \dot{w}_p &= \frac{-w_p(R_a J + L_a B) - \omega(R_a B + K_i K_m) + V_a K_i}{J L_a} - \frac{T_L}{J} \\
    \dot{i}_a &= \frac{-R_a i_a - K_m \omega + V_a}{L_a} \\
    \dot{\theta} &= \omega
\end{align*}

\noindent Estas ecuaciones fueron discretizadas e integradas en el archivo \texttt{modmotor.m}, mediante una estructura de la forma:

\begin{equation*}
    x(k+1) = x(k) + h \cdot \dot{x}(k)
\end{equation*}

Inicialmente, se empleó el mismo paso de integración \(h = 10^{-7}\,\text{s}\) utilizado en simulaciones anteriores con un conjunto distinto de parámetros del motor. Sin embargo, al utilizar los parámetros ajustados recientemente, se observó que el sistema se volvía inestable o divergía numéricamente, arrojando salidas de orden \(10^{300}\). Esto indicó un problema de estabilidad numérica del método de Euler debido a la presencia de polos más lentos en el sistema.

Se procedió entonces a calcular el polo dominante del sistema a partir de su función de transferencia, que resultó ser:

\begin{equation*}
    G(s) = \frac{K}{0.01432\,s^2 + 0.346\,s + 1}
\end{equation*}

Los polos asociados a esta dinámica se ubican aproximadamente en \(s \approx -17.9\) y \(s \approx -4.0\), por lo que la constante de tiempo más rápida es del orden de \(\tau \approx 1/17.9 \approx 0.056\,\text{s}\). Se utilizó como criterio de estabilidad para el método de Euler:

\begin{equation*}
    h < \frac{2}{|\text{Re}(p_{\text{más rápido}})|} \approx \frac{2}{17.9} \approx 0.112\,\text{s}
\end{equation*}

Por lo tanto, se seleccionó un paso de integración \(h = 10^{-2}\,\text{s}\), que resulta suficientemente pequeño como para capturar la dinámica del sistema sin comprometer la estabilidad numérica, y suficientemente grande como para evitar errores acumulativos por redondeo.

El controlador PID fue discretizado mediante una fórmula recursiva directa:

\begin{equation*}
    u(k) = u(k-1) + A_1 e(k) + B_1 e(k-1) + C_1 e(k-2)
\end{equation*}

donde los coeficientes fueron obtenidos a partir de la discretización por diferencias progresivas:

\begin{align*}
    A_1 &= \frac{2K_p T_s + K_i T_s^2 + 2K_d}{2T_s} \\
    B_1 &= \frac{-2K_p T_s + K_i T_s^2 - 4K_d}{2T_s} \\
    C_1 &= \frac{K_d}{T_s}
\end{align*}

\noindent Los valores iniciales sugeridos fueron \(K_p = 0.1\), \(K_i = 0.01\) y \(K_d = 5\), que permitieron observar un comportamiento transitorio aceptable y una buena compensación ante la aparición del torque de carga \(T_L(t)\). La acción de control fue limitada a \(\pm12\,\text{V}\) mediante saturación, tal como ocurre en una implementación práctica.

\subsection*{Conclusión}

Se logró implementar correctamente un control PID en tiempo discreto para regular el ángulo del motor ante perturbaciones de carga. El paso de integración del método de Euler fue ajustado a \(h = 10^{-2}\,\text{s}\) para garantizar la estabilidad numérica, dado el nuevo conjunto de parámetros del motor. La acción de control permitió alcanzar y mantener la referencia angular deseada, incluso en presencia de cargas externas variables, demostrando la robustez del diseño PID frente a perturbaciones.


\end{document}
